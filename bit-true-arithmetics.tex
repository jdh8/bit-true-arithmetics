\documentclass{beamer}
\usepackage[no-math]{fontspec}
\usepackage{xeCJK}
\hypersetup{colorlinks,linkcolor=}

\usetheme{CambridgeUS}
\title{Bit-true arithmetics}
\author[jdh8]{何震邦 (Chen-Pang He, jdh8)}
\date{March 19, 2025}
\institute{Skymizer}

\begin{document}
\maketitle

\begin{frame}{Quantization}
    \begin{itemize}
        \item Quantization maps a large/continuous set to a small/countable set.
        \item Quantization is required because we only have finite bits to store data.
        \item Further quantization reduces space and probably time usage.
    \end{itemize}
\end{frame}

\begin{frame}{Accuracy and precision}
    \begin{figure}
        \caption{
            \href
                {https://en.wikipedia.org/wiki/Accuracy_and_precision\#ISO_definition_(ISO_5725)}
                {Accuracy according to BIPM and ISO 5725}
        }
        \begin{minipage}{0.4\textwidth}
            \centering
            \includegraphics[width=0.8\textwidth]{assets/High_accuracy_Low_precision.pdf} \\
            Low accuracy due to low precision
        \end{minipage}
        \begin{minipage}{0.4\textwidth}
            \centering
            \includegraphics[width=0.8\textwidth]{assets/High_precision_Low_accuracy.pdf} \\
            Low accuracy despite of high precision
        \end{minipage}
    \end{figure}
\end{frame}

\begin{frame}{Rounding mode}
    \begin{figure}
        \href
            {https://upload.wikimedia.org/wikipedia/commons/8/8a/Comparison_rounding_graphs_SMIL.svg}
            {\includegraphics[height=0.625\textheight]{assets/Comparison_rounding_graphs_SMIL.pdf}}

        \caption{
            \href{https://upload.wikimedia.org/wikipedia/commons/8/8a/Comparison_rounding_graphs_SMIL.svg}{Interactive graph}
            by CMG Lee on
            \href{https://commons.wikimedia.org/wiki/File:Comparison_rounding_graphs_SMIL.svg}{Wikimedia Commons}
        }
    \end{figure}
\end{frame}

\begin{frame}{Arithmetics}
    Arithmetic results can be inexact in the target format.
    
    \begin{example}
        Consider 0.1 + 0.2 in IEEE 754 binary64 format.

        \begin{itemize}
            \item 0.1 $\approx$ \texttt{0x1.999999999999ap-4}
            \item 0.2 $\approx$ \texttt{0x1.999999999999ap-3}
            \item 0.3 $\approx$ \texttt{0x1.3333333333333p-2}
            \item \texttt{0x1.999999999999ap-4} + \texttt{0x1.999999999999ap-3}
                \begin{itemize}
                    \item[] = \texttt{0x1.3333333333334p-2}
                    \item[] $\approx$ 0.30000000000000004
                \end{itemize}
        \end{itemize}
    \end{example}
\end{frame}

\begin{frame}{Double rounding problem}
    \begin{itemize}
        \item Rounding is not associative!
        \item 9.46 $\to$ 9
        \item 9.46 $\to$ 9.5 $\to$ 10
    \end{itemize}
\end{frame}

\end{document}